\PassOptionsToPackage{table,svgnames,dvipsnames}{xcolor}

\usepackage[utf8]{inputenc}
\usepackage[T1]{fontenc}
\usepackage[sc]{mathpazo}
\usepackage[ngerman,english]{babel} % english is the same as american or USenglish
\usepackage[autostyle]{csquotes}
\usepackage[%
  backend=biber,
  url=true,
  style=apa, % alphabetic, numeric
  sorting=none, % default == nty, https://tex.stackexchange.com/questions/51434/biblatex-citation-order
  maxnames=4,
  minnames=3,
  maxbibnames=99,
  giveninits,
  uniquename=init]{biblatex} % TODO: adapt citation style
\usepackage{graphicx}
\usepackage{svg}
\usepackage{scrhack} % necessary for listings package
\usepackage{listings}
\usepackage{lstautogobble}
\usepackage{tikz}
\usepackage{pgfplots}
\usepackage{pgfplotstable}
\usepackage{booktabs} % for better looking table creations, but bad with vertical lines by design (package creator despises vertical lines)
\usepackage[final]{microtype}
\usepackage{caption}
\usepackage[hidelinks]{hyperref} % hidelinks removes colored boxes around references and links
\usepackage{ifthen} % for comparison of the current language and changing of the thesis layout
\usepackage{pdftexcmds} % string compare to work with all engines
\usepackage{paralist} % for condensed enumerations or lists
\usepackage{subfig} % for having figures side by side
\usepackage{siunitx} % for physical accurate units and other numerical presentations
% \usepackage{multirow} % makes it possible to have bigger cells over multiple rows in a table
\usepackage{array} % different options for table cell orientation
% \usepackage{makecell} % allows nice manual configuration of cells with linebreaks in \thead and \makecell with alignments
% \usepackage{pdfpages} % for including multiple pages of pdfs
% \usepackage{adjustbox} % can center content wider than the \textwidth
% \usepackage{tablefootnote} % for footnotes in tables as \tablefootnote
% \usepackage{threeparttable} % another way to add footnotes as \tablenotes with \item [x] <your footnote> after setting \tnote{x} 
\usepackage{tabularx}

% https://tex.stackexchange.com/questions/42619/x-mark-to-match-checkmark
\usepackage{amssymb}% http://ctan.org/pkg/amssymb
\usepackage{pifont}% http://ctan.org/pkg/pifont
\newcommand{\cmark}{\ding{51}}%
\newcommand{\xmark}{\ding{55}}%


\usepackage[acronym,xindy,toc]{glossaries} % TODO: include "acronym" if glossary and acronym should be separated
\makeglossaries
% refer to https://en.wikibooks.org/wiki/LaTeX/Glossary for acronyms and glossary entries

\newacronym[shortplural={VCs}, description={Verifiable Credentials: a W3C Recommendation for digital credentials \parencite{Sporny.18Kas2019}, also refers to a verifiable credential which is a document complying the specification}]{VC}{VC}{Verifiable Credentials}
\newacronym{W3C}{W3C}{The World Wide Web Consortium}
\newacronym{JSON}{JSON}{Javascript Object Notation}
\newacronym{JSON-LD}{JSON-LD}{\acrshort{JSON} Linked Data}
\newacronym{JWT}{JWT}{\acrshort{JSON} Web Token}
\newacronym{XML}{XML}{Extensible Markup Language}
\newacronym{YAML}{YAML}{YAML Ain't Markup Language}
\newacronym{zk-proofs}{zk-proofs}{Zero-Knowledge Proofs}
\newacronym{ZKP}{ZKP}{Zero-Knowledge Proofs}
\newacronym{IRI}{IRI}{Internationalized Resource Identifier}
\newacronym{URI}{URI}{Uniform Resource Identifier}
\newacronym{URL}{URL}{Uniform Resource Locator}
\newacronym{URN}{URN}{Uniform Resource Name}
\newacronym{RFC}{RFC}{Request For Comments}
\newacronym{RDF}{RDF}{Rescource Description Framework}
\newacronym{GDPR}{GDPR}{General Data Protection Regulation}
\newacronym{CL}{CL}{Camenisch-Lysyanskaya (signatures)}
\newacronym{DID}{DID}{Decentralized Identifier}
\newacronym{SSI}{SSI}{Self-Sovereign Identity}
\newacronym{IoT}{IoT}{Internet of Things}
\newacronym{HTTP}{HTTP}{Hypertext Transfer Protocol}
\newacronym{DLT}{DLT}{Distributed Ledger Technology} % important update for glossaries, before document


\bibliography{bibliography}

\setkomafont{disposition}{\normalfont\bfseries} % use serif font for headings
\linespread{1.25} % adjust line spread for mathpazo font

% Add table of contents to PDF bookmarks
\BeforeTOCHead[toc]{{\cleardoublepage\pdfbookmark[0]{\contentsname}{toc}}}

% Define TUM corporate design colors
% Taken from http://portal.mytum.de/corporatedesign/index_print/vorlagen/index_farben
\definecolor{TUMBlue}{HTML}{0065BD}
\definecolor{TUMSecondaryBlue}{HTML}{005293}
\definecolor{TUMSecondaryBlue2}{HTML}{003359}
\definecolor{TUMBlack}{HTML}{000000}
\definecolor{TUMWhite}{HTML}{FFFFFF}
\definecolor{TUMDarkGray}{HTML}{333333}
\definecolor{TUMGray}{HTML}{808080}
\definecolor{TUMLightGray}{HTML}{CCCCC6}
\definecolor{TUMAccentGray}{HTML}{DAD7CB}
\definecolor{TUMAccentOrange}{HTML}{E37222}
\definecolor{TUMAccentGreen}{HTML}{A2AD00}
\definecolor{TUMAccentLightBlue}{HTML}{98C6EA}
\definecolor{TUMAccentBlue}{HTML}{64A0C8}

% Settings for pgfplots
\pgfplotsset{compat=newest}
\pgfplotsset{
  % For available color names, see http://www.latextemplates.com/svgnames-colors
  cycle list={TUMBlue\\TUMAccentOrange\\TUMAccentGreen\\TUMSecondaryBlue2\\TUMDarkGray\\},
}

% Settings for lstlistings

\definecolor{lightgray}{rgb}{0.95, 0.95, 0.95}
\definecolor{darkgray}{rgb}{0.4, 0.4, 0.4}
%\definecolor{purple}{rgb}{0.65, 0.12, 0.82}
\definecolor{editorGray}{rgb}{0.95, 0.95, 0.95}
\definecolor{editorOcher}{rgb}{1, 0.5, 0} % #FF7F00 -> rgb(239, 169, 0)
\definecolor{editorGreen}{rgb}{0, 0.5, 0} % #007C00 -> rgb(0, 124, 0)
\definecolor{orange}{rgb}{1,0.45,0.13}		
\definecolor{olive}{rgb}{0.17,0.59,0.20}
\definecolor{brown}{rgb}{0.69,0.31,0.31}
\definecolor{purple}{rgb}{0.38,0.18,0.81}
\definecolor{lightblue}{rgb}{0.1,0.57,0.7}
\definecolor{lightred}{rgb}{1,0.4,0.5}
\usepackage{upquote}
\usepackage{listings}
% CSS
\lstdefinelanguage{CSS}{
  keywords={color,background-image:,margin,padding,font,weight,display,position,top,left,right,bottom,list,style,border,size,white,space,min,width, transition:, transform:, transition-property, transition-duration, transition-timing-function},	
  sensitive=true,
  morecomment=[l]{//},
  morecomment=[s]{/*}{*/},
  morestring=[b]',
  morestring=[b]",
  alsoletter={:},
  alsodigit={-}
}

% JavaScript
\lstdefinelanguage{JavaScript}{
  morekeywords={typeof, new, true, false, catch, function, return, null, catch, switch, var, if, in, while, do, else, case, break},
  morecomment=[s]{/*}{*/},
  morecomment=[l]//,
  morestring=[b]",
  morestring=[b]'
}

\lstdefinelanguage{HTML5}{
  language=html,
  sensitive=true,	
  alsoletter={<>=-},	
  morecomment=[s]{<!-}{-->},
  tag=[s],
  otherkeywords={
  % General
  >,
  % Standard tags
	<!DOCTYPE,
  </html, <html, <head, <title, </title, <style, </style, <link, </head, <meta, />,
	% body
	</body, <body,
	% Divs
	</div, <div, </div>, 
	% Paragraphs
	</p, <p, </p>,
	% scripts
	</script, <script,
  % More tags...
  <canvas, /canvas>, <svg, <rect, <animateTransform, </rect>, </svg>, <video, <source, <iframe, </iframe>, </video>, <image, </image>, <header, </header, <article, </article
  },
  ndkeywords={
  % General
  =,
  % HTML attributes
  charset=, src=, id=, width=, height=, style=, type=, rel=, href=,
  % SVG attributes
  fill=, attributeName=, begin=, dur=, from=, to=, poster=, controls=, x=, y=, repeatCount=, xlink:href=,
  % properties
  margin:, padding:, background-image:, border:, top:, left:, position:, width:, height:, margin-top:, margin-bottom:, font-size:, line-height:,
	% CSS3 properties
  transform:, -moz-transform:, -webkit-transform:,
  animation:, -webkit-animation:,
  transition:,  transition-duration:, transition-property:, transition-timing-function:,
  }
}

\lstdefinestyle{htmlcssjs} {%
  % General design
%  backgroundcolor=\color{editorGray},
  basicstyle={\footnotesize\ttfamily},   
  frame=b,
  % line-numbers
  xleftmargin={0.75cm},
  numbers=left,
  stepnumber=1,
  firstnumber=1,
  numberfirstline=true,	
  % Code design
  identifierstyle=\color{black},
  keywordstyle=\color{blue}\bfseries,
  ndkeywordstyle=\color{editorGreen}\bfseries,
  stringstyle=\color{editorOcher}\ttfamily,
  commentstyle=\color{brown}\ttfamily,
  % Code
  language=HTML5,
  alsolanguage=JavaScript,
  alsodigit={.:;},	
  tabsize=2,
  showtabs=false,
  showspaces=false,
  showstringspaces=false,
  extendedchars=true,
  breaklines=true,
  % German umlauts
  literate=%
  {Ö}{{\"O}}1
  {Ä}{{\"A}}1
  {Ü}{{\"U}}1
  {ß}{{\ss}}1
  {ü}{{\"u}}1
  {ä}{{\"a}}1
  {ö}{{\"o}}1
}

\definecolor{eclipseStrings}{RGB}{42,0.0,255}
\definecolor{eclipseKeywords}{RGB}{127,0,85}
\colorlet{numb}{magenta!60!black}

\lstdefinelanguage{json}{
    basicstyle=\normalfont\ttfamily,
    commentstyle=\color{eclipseStrings}, % style of comment
    stringstyle=\color{eclipseKeywords}, % style of strings
    numbers=left,
    numberstyle=\scriptsize,
    stepnumber=1,
    numbersep=8pt,
    showstringspaces=false,
    breaklines=true,
    breakatwhitespace=false,
    postbreak=\mbox{\textcolor{red}{$\hookrightarrow$}\space},
    frame=lines,
    backgroundcolor=\color{editorGray}, %only if you like
    string=[s]{"}{"},
    comment=[l]{:\ "},
    morecomment=[l]{:"},
    literate=
        *{0}{{{\color{numb}0}}}{1}
         {1}{{{\color{numb}1}}}{1}
         {2}{{{\color{numb}2}}}{1}
         {3}{{{\color{numb}3}}}{1}
         {4}{{{\color{numb}4}}}{1}
         {5}{{{\color{numb}5}}}{1}
         {6}{{{\color{numb}6}}}{1}
         {7}{{{\color{numb}7}}}{1}
         {8}{{{\color{numb}8}}}{1}
         {9}{{{\color{numb}9}}}{1}
}

% Use this for basic highlighting
\lstset{%
  basicstyle=\ttfamily,
  columns=fullflexible,
  autogobble,
  keywordstyle=\bfseries\color{TUMBlue},
  stringstyle=\color{TUMAccentGreen}
}

% Settings for search order of pictures
\graphicspath{
    {logos/}
    {figures/}
}

% Set up hyphenation rules for the language package when mistakes happen
\babelhyphenation[english]{
an-oth-er
ex-am-ple
}

% Decide between
%\newcommand{\todo}[1]{\textbf{\textsc{\textcolor{TUMAccentOrange}{(TODO: #1)}}}} % for one paragraph, otherwise error!
%\newcommand{\done}[1]{\textit{\textsc{\textcolor{TUMAccentBlue}{(Done: #1)}}}} % for one paragraph, otherwise error!
% and
\newcommand{\todo}[1]{{\bfseries{\scshape{\color{TUMAccentOrange}[(TODO: #1)]}}}} % for multiple paragraphs
\newcommand{\done}[1]{{\itshape{\scshape{\color{TUMAccentBlue}[(Done: #1)]}}}} % for multiple paragraphs
% for error handling of intended behavior in your latex documents.

\newcommand{\tabitem}{~~\llap{\textbullet}~~}

\newcolumntype{P}[1]{>{\centering\arraybackslash}p{#1}} % for horizontal alignment with limited column width
\newcolumntype{M}[1]{>{\centering\arraybackslash}m{#1}} % for horizontal and vertical alignment with limited column width
\newcolumntype{L}[1]{>{\raggedright\arraybackslash}m{#1}} % for vertical alignment left with limited column width
\newcolumntype{R}[1]{>{\raggedleft\arraybackslash}m{#1}} % for vertical alignment right with limited column width