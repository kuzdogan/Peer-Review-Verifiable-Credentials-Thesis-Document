% !TeX root = ../main.tex
% Add the above to each chapter to make compiling the PDF easier in some editors.

\chapter{Conclusion}\label{chapter:conclusion}

In this work, a design science research on peer review recognition was presented and a system design was delivered as an artifact. The work initially gave an overview of the peer review. The problems discussed in the literature were introduced, and the currently misaligned incentives were highlighted, that is to publish more and get cited more instead of doing peer reviews. Despite being considered highly important and essential in scientific knowledge generation, scientists are not incentivized to do peer reviews and there are concerns on the implications of this on the quality and the sustainability of the peer review. Further, the problem was investigated and the lack of recognition for the peer review work of the researchers was described. It was argued that this is partly caused by the nature of the practice of blinded reviewing, that requires identities to be hidden and reviews not to be published. The existing platforms were examined that are tackling this problem and current and future problems were identified from an open science perspective. It was observed that these platforms aggregate data by acting as trusted third-party peer review verifiers. From this specific problem the following research question was derived: "How can closed peer reviews be verified without trusted third-parties?"

To answer this question a system using \acrlong{VC} and \acrlong{zk-proofs} was designed and developed. The designed system enables the verification of peer reviews through journals issuing peer review credentials. It also includes an open review showcasing platform where reviewers can add these verifiable credentials without breaking the anonymity of the blinded reviews. A proof of concept prototype was also implemented to show the technical feasibility of the conceptual design.

The evaluation shows how the self-imposed requirements were fulfilled but also demonstrates the need for further validating the significance of the problem. The insights gained in the interviews would shed light on peer review research and future work. 


Overall, this work is a first step in the unexplored space of peer review verification and recognition in the literature. It presents a system that enables the verification of peer reviews without a trusted third-party. It also demonstrates how the state of art standards and zero-knowledge cryptography can be used to benefit peer reviewers. The evaluation suggests further investigation is required in the problem space and the attitudes of researchers towards showcasing reviews. The designed system answers the research question proposed. With the design of the system, several learnings were gained. First, with the current significance of peer review records from researchers' perspective, it is crucial for such a showcasing system to attain minimum user input. Furthermore, there need to be increasing institutional demand for the review records and the system to be perceived useful. Familiarity with open science practices is also found to increase the perceived usefulness of the system. With these variables found, an acceptance model was proposed. It was observed that the designed system also inherits the usability problems currently existing in the \acrshort{SSI} applications although some of those could be avoided by using alternative centralized identifiers. This system design with \acrshort{VC} and \acrshort{zk-proofs} is among few in literature and serves as a reference for future implementations and designs. It also contributes as a system to the larger peer review incentivization problem, and brings light into the potential implications of the current peer review showcasing systems. 

Future work can also include the validation of the proposed acceptance model to better understand user behaviour and the potential system usage. Following implementations would benefit from working with real-world publishers and making use of real world data to see how the designed system would adopt, and what problems may arise. It would also uncover how and if the automatization of adding the peer review records can be achieved, which is found to be the main barrier for the user adoption.